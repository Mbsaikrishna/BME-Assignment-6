\documentclass[12pt]{article}

\begin{document}

\section*{Introduction}

Due to the specific obstacles highlighted by epidemiologists, immunologists, and medical practitioners, biomedical science and engineering have been proposed as promising areas to assist medical science in combating SARS-CoV-2.Survival, symptomology, protein surface composition, and infection pathways are all factors to consider.Design is based on these multidisciplinary technical concepts.to research and create techniques of prevention, diagnoses, monitoring, and therapies.

\section{Disinfecting N-95}

The Ministry of Science and Technology announced on Friday that a group of Indian scientists has created a self-disinfecting antiviral mask to combat the Covid-19 outbreak.

According to the ministry, the antiviral mask coated with copper-based nanoparticles performs effectively against coronavirus and other viral and bacterial illnesses.

The mask is biodegradable, washable, and highly breathable.

Because the virus is transmitted mostly by respiratory particles that are airborne, wearing a mask has been one of the most significant and successful health precautions to contain it. The ministry did highlight, however, that preventing viral transmission with traditional masks has proven challenging, particularly in densely populated areas such as hospitals, airports, train stations, and shopping malls, where the virus load is quite high.

\section{3-D printing Mask}

We show that employing a low-cost 3D-printed frame to improve the fit of a fabric mask improves its inward protection efficacy for airborne particles known to transmit SARS-CoV-2. In comparison to the effectiveness attained in the absence of the frame, a 3D-printed flexible frame (i.e., brace) boosted the inward protection efficiency of a cotton-based fabric mask by 13–43 percent for particles ranging in size from 0.5–2 m. The introduction of a flexible form-fitting frame, for example, boosted inward protection efficacy by 31 and 40 percent for 0.5 and 1 m particles, respectively. For optimising the facial contact area and mechanical matching, rapid prototyping of the mask frame geometry and material properties was also highlighted

\section{Surface Decontamination}

In addition to recommended PPE for caregivers and patients, hand hygiene and
surface decontamination are also key to health safety, as the SARS-CoV-2 is known
to remain viable on surfaces for hours to days. Disinfection practices should be
focused on meticulous hygiene in workspaces to minimize contaminated surfaces.
As mentioned previously, the nosocomial spread has been documented for infectious
pathogens, including coronaviruses. The multifunctionality of nanomaterials endow
them with the feasibility of being coated on a large number of common In addition
to being cost-effective and easy to synthesize, copper oxide (CuO) NPs exhibit 
interesting biological properties. For these reasons, they have been incorporated
successfully for microorganism and virus inactivation purposes on contaminated sur-
faces. Thus, CuO NPs will play an important role to promote the risk reduction
of people exposed to the COVID-19 virus spread. surfaces to act as biocidal agents for
different types of pathogens, including viruses

\section{ Continuous Positive Airway Pressure (CPAP}

The next step up in treating COVID-19 patients is Continuous Positive
Airway Pressure (CPAP) which is initially intended to prevent airways
collapse in sleep apnoea patients, but has been shown to be beneficial to
COVID patients if applied early enough in the progression of the disease.
A well-fitted face mask is an essential component of a CPAP system and
as such it is quite intrusive. CPAP is only appropriate for patients who
are capable of some breathing strength as CPAP effectively opposes some
resistance to expiration. Variants exist that either automatically adjust the
level of pressure to the patients breathing characteristics (APAP) or have
different levels of pressure for inspiration and expiration (BiPAP). CPAP
usually supplies (filtered) air to the patient but most masks have a port for
supplementing the supply with oxygen.




\section{VENTILATORS}


Patients who are unable to breathe on their own must be placed on a ventilator.
Patients in an advanced stage of respiratory distress are frequently intubated
and sedated at the start of treatment since ventilators can replace breath function.
Patients in an advanced stage of respiratory distress are frequently intubated and sedated at the start of treatment since ventilators can replace breath
function. They are complicated devices that give healthcare providers a lot of
flexibility in terms of adjusting assisted breathing settings and eventually weaning healing patients off the ventilator. Modern ventilators are usually pressurecontrolled closed loops that can detect spontaneous breathing and provide synchronised aid to recovering patients. They also allow the patient to alter the
composition of the gas he or she breathes, ranging from regular air to 100 percent
oxygen. They normally get their supply from the hospital’s gas supply network,
but they can also be connected to oxygen tanks or oxygen concentrators if there
isn’t one.






\end{document}
